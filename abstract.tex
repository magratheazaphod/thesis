% (This file is included by thesis.tex; you do not latex it by itself.)

\begin{abstract}

% The text of the abstract goes here.  If you need to use a \section
% command you will need to use \section*, \subsection*, etc. so that
% you don't get any numbering.  You probably won't be using any of
% these commands in the abstract anyway.

The Asian summer monsoon supplies around 3 billion people with much of their yearly supply of freshwater, necessary for human consumption as well as in agriculture and industry. In many regions, particularly along the Ganges River in India and in northern China, use of freshwater far exceeds natural recharge rates. Given the high population density of these regions, a substantial fraction of Asia’s population is therefore critically sensitive to changes in the supply of freshwater by the monsoon under 21st century warming. A first step toward future projection is to consider the available observational record of rainfall. This dissertation quantifies the spatial and temporal variability of the present-day Asian summer monsoon, focusing on two major subsystems: the Indian (or South Asian) monsoon, and the East Asian monsoon.

In summer, two distinct rainfall régimes are observed: Convective storms over India, Bangladesh and Nepal (the South Asian monsoon), and frontal rainfall over China, Japan and Korea (the East Asian monsoon). In addition, the Himalayas and other orography, including the Arakan Mountains, Ghats and Yunnan Plateau, create smaller precipitation domains separated by sharp gradients. My research has revealed a previously unrecognized mode of continental precipitation variability that spans both South and East Asia during July and August. A dipole between the Himalayan Foothills and the “Monsoon Zone” of central India dominates July-August interannual variability in South Asia, and is also associated in East Asia with a tripole between the Yangtze Corridor and North and South China. By performing lag-lead correlation of rainfall, I show that this covariation does not correspond to the spatial pattern of July-August storm tracks. Instead, I hypothesize that interannual change in the strength of moisture transport from the Bay of Bengal to the Yangtze Corridor across the northern Yunnan Plateau induces widespread precipitation anomalies. Abundant moisture transport along this route requires both cyclonic monsoon circulation over India and sufficient heating over the Bay of Bengal, which occurs only during July and August, as I also show by analyzing existing runs with a zoomed, nested version of the LMDZ5 model mudged to reanalysis.

In China, a growing body of work identifies a régime shift in rainfall occurring in the late 1970s known as the “South Flood-North Drought.” However, this phenomenon has not previously been described in terms of the complex seasonal cycle of the East Asian monsoon. During the peak interval of rainfall, known as Meiyu season, a persistent but meandering front (the “Meiyu Front”) delivers strong bouts of rainfall over a narrow latitude band. The preferred latitude of this front shifts throughout the year and shifts abruptly from May to July. I have developed an image processing algorithm that produces a 57-year (1951-2007) daily catalog of frontal rainfall events in China, reporting its latitude, intensity and coherence. This result allows for a quantitative assessment of Meiyu change between 1951-1979 and 1980-2007, which has never previously been performed on an event-by-event basis. I find that the greatest change in frontal behavior has occurred during May, when events have strongly decreased over the Yangtze River valley. In addition, a statistically significant southward shift has occurred in July-August Meiyu events. Both of these months have also witnessed a southward displacement in the latitude of the tropospheric jet toward the end of the 20th century. Thus, I propose the novel hypothesis that the “South Flood-North Drought” can be attributed to a change in the timing and extent of the tropospheric jet’s meridional transit.

Finally, I revisit the previously described hypothesis of moisture transport across the Yunnan Plateau as a key component of interannual variability. I assemble a suite of runs with the LMDZ5 model, which features a zoomed nested grid over the Tibetan Plateau and has previously been found to perform well in the region of interest. Idealized experiments are performed, including not only the lowering of the Tibetan Plateau and Himalayan Foothills, but also the raising or elimination of the Yunnan Plateau, the imposition of positive and negative heating anomalies over the “Monsoon Zone” and changing the specific humidity of air parcel. In addition, I test whether the injection or removal of water vapor in the Sichuan region can induce the zonal tripole pattern of rainfall response found in observation. We also verify the potential contribution of coupling across the Yunnan Plateau to the “South Flood-North Drought” by tracking shifts in the position of the tropospheric jet in each run. In observation, global-scale anomalies of the jet are associated with our leading mode of Asian monsoon variability, a behavior replicated in the model. Finally, I propose that 21st century changes in rainfall in South and East Asia may be anticipated by considering the change in moisture transport from the Bay of Bengal to the Yangtze Corridor.


\end{abstract}
