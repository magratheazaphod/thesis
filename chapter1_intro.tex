\chapter{Introduction}

\blockquote{
%
%	``On the day the monsoon began, I was swimming in the river with a dozen other young men and about twenty children. The dark clouds, which had painted their sombre moods on the sky for weeks, gathered from horizon to horizon, and seemed to press upon the tops of the tallest trees. The air, after eight dry months, was so lavishly perfumed with rain that we were almost drunk with excitement.
%	
%	`Paous alla! S'alla ghurree!' the children cried repeatedly, grasping my hands. They pointed to the clouds and dragged me toward the village. `The rain is coming! Let's go home!'
%	
%	The first drops of rain fell as we ran. In seconds, the drops were a heavy fall. In minutes, the fall was a cascade. Within an hour, the monsoon was a ceaseless torrent, so thick that it was difficult to breathe in the open without cupping my hands to my mouth to make a little cave of air.
%	
%	At first, the villagers danced in the rain and played pranks on one another. Some took soap, and washed in the heaven-sent shower. Some went to the local temple, where they knelt in the rain to pray. Others busied themselves with repairs to the roofs of their houses and the drainage trenches dug around every mud-brick wall.
%	
%	Eventually, everyone stopped to simply stare at the drifting, flapping, curling sheets of rain. Every doorway of every house was crowded with faces, and each flash of lightning showed the frozen tableaux of wonder.
%	
%	That downpour of several hours was followed by a lull just as long. The sun shone intermittently, and rainwater steamed from the warming earth. The first ten days of the season proceeded in the same way, with violent storms and tranquil lulls, as if the monsoon was probing the village for its weaknesses before mounting a final assault.
%	
%	Then, when the great rain came, it was a lake of water in the air, and it rained almost without pause for seven days and nights. On the seventh day, I was at the river's edge, washing my few clothes as the drenching torrents fell. At one point I reached for my soap, and realised that the rock I'd place it on was submerged. The water, which had merely caressed my bare feet, rose from my ankles to my knees in seconds. As I looked upstream at the tumbling crash of the river, the water reached to my thighs, and was still rising.
	
	`The river! The river is coming!' I shouted, in broken Marathi.

	Sensing my distress but not really understanding me, the villagers gathered around and then called Prabaker, plying him with questions.
	
	`What is your matter, Lin? The people are very upset for you.'
	
	`The river! It's coming up fast. It'll wipe the village out!'
	
	Prabaker smiled.
	
	`Oh, no, Lin. That will not be happening.'
	
	`I'm telling you! I've seen it. I'm not joking, Prabu. The fucking river's in flood!'
	
	Prabaker translated my words for the others. Everyone laughed.
	
%	`Are you all crazy?' I shouted, in exasperation. `It's not funny!'
%	
%	They laughed all the harder and crowded around me, reaching out to calm my fear by patting and stroking me, their laughing voices full of soothing words and sighs. Then, with Prabaker leading the way, the crowd of villagers goaded, dragged and pushed me toward the river.
	
	The river, only a few hundred meters away, was a deluge: a vast muddy concrescence that tore through the valley in heaving waves and boiling eddies. The rain redoubled its intensity as we stood there, our clothes as drenched as the yielding soil. And still the tumid river grew, consuming new land with each thumping heartbeat.
	
	`You see those sticks, Lin,' Prabaker said, in his most irritating attempt at a soothing tone. `Those sticks are the flood-game sticks. Do you remember, when the people put them in the ground? Satish and Pandey, Narayan and Bharat...do you remember?'
	
%	I did remember. Days before, there'd been a lottery of some kind ... 
%	
%	One hundred and twelve numbers - one for every man in the village - were written on small pieces of paper, and mixed together in an empty clay water-pot, called a matka. The men lined up to draw their numbers, and then a second set of the same numbers was mixed in the pot. A little girl was given the honour of drawing the six winning numbers from the pot. The whole village watched the ceremony, and applauded the winners happily.
%	
%	The six men whose numbers had been drawn had won the chance to hammer a wooden stake, a little over a metre long, into the earth. As well, the three oldest men in the village were accorded the right to a wooden stake without the numbered lottery. They duly chose places for their stakes, and younger men obliged by hammering the wooden pegs into the ground. When all nine stakes were positioned, little flags with the names of the men were tied to each one, and the people drifted back to their homes.
%	
%	I'd watched the affair from a shady spot beneath the branched dome of a tree. At the time, I was working on my own small reference dictionary of the Marathi language, based on phonetic spellings of the words I heard every day in the village. I gave the ceremony little attention, and I never bothered to ask its purpose.
	
	I did remember. Days before, there'd been a lottery of some kind ... As we stood in the numbing, drumming rain and watched the prowling advance of the river, Prabaker explained that the wooden stakes were part of a flood-game that was played every year. The oldest men in the village, and six lottery winners, were given the chance to predict the point to which the river would rise. Each wooden stick, with its flag of yellow silk, represented a best guess.
	
	`You see, this one little flag?' Prabaker asked, pointing to the stake that was furthest from where we stood. `This one is almost gone. The river will reach to him, and cover him, tomorrow or tonight.' 
	
	He translated what he'd told me for the crowd, and they pushed Satish, a heavy-set cowherd, to the front of the group. The almost submerged stick was his, and he accepted, with shy laughter and downcast eyes, the good-natured jeers of his friends and the sneers of the older men.
	
	`And this one here,' Prabaker went on, pointing to the stake nearest to our position. `This one is the river will never be touching. The river never comes more far than this place. Old Deepakbhai has picked for himself this place, for the putting of his stick. He thinks this year will be a very heavy monsoon.'
	
%	The villagers had lost interest, and were already drifting or jogging back to the village. Prabaker and I stood alone.
	
	`But ... how do you know that the river won't rise past this point?'
	
	`We are here a long time, Lin. Sunder village has been in this place for two thousands of years. The next village, Natinkerra, has been there for much longer, about three thousands of years. In some other places - not near to here - the people do have a bad experiences, with the floods, in monsoon time. But not here. Not in Sunder. Our river has never come to this far. This year, also, I don't think it will come to this far, even so old Deepakbhai says it will. Everybody knows where the river will stop, Lin.' \attrib{Gregory David Roberts, \textit{Shantaram}, 133-135}}
	
	Clearly, atmospheric dynamicists are not the only ones to engage in the art of monsoon prognostication. Nonetheless, in spite of a century of work, forecasts of present-day monsoon seasonal variability and of future changes under global warming remain nebulous. The purpose of this dissertation is to elucidate the atmospheric dynamics implicated with the leading mode of Asian summer monsoon variability and reveal connections to other important modes of climate variability. The ultimate vision is that increasingly skillful projections of these other climate components will also be used to improve projection of changes in the \nth{21} century monsoon.

\section{A characterization}

An extensive description of the current dynamic theory of the monsoon is presented in Chapter 2. Instead, we focus here on the importance of the monsoon to the global hydrological circulation and the magnitude of the human impacts of its variations.

A monsoonal climatology is typified by the summer arrival of consistent, heavy rainfall, and relatively dry conditions during the rest of the year. With its yearly arrival have followed centuries of attempts to prognosticate its advance and severity. Its outcome is linked to rate changes in Indian Gross Domestic Production (GDP) \citep{Gadgil}

The Asian monsoon is dominated by the presence of the Tibetan Plateau, the largest topographic barrier on earth. Standing at over 5,000 meters, the Tibetan Plateau has been the dominant geological feature on Earth for the past 10 million years, and possibly as many as 50 million years. The region therefore functions as a sort of fluid dynamical laboratory whose properties are still not well understood. The earliest modeling studies immediately began to point to the Tibetan Plateau as a potential dominant control of monsoon strength \cite{Manabe}

The underlying motivation for the study of rainfall variability lies in the magnitude of human impacts. On an hourly to daily time scale, extreme rainfall events inflict massive damage and suffering on the Asian Monsoon region. This is borne out culturally by traditions such as the Thai river festival and even the history of Chinese science, wherein sophisticated engineering techniques were developed early in Chinese history for attempting to control the Yangtze River, and culminating in the creation of the Three Gorges Dam. On a monthly time scale, rainfall supply dictates agricultural yields and even long-term demography. In considering the changes in rainfall due to global warming, both shifts in mean and in the tails of the distribution need to be characterized \cite{Pendergrass}.


\section{Rainfall variability}

\subsection{Statistical Properties and Heterogeneity}

The study of rainfall variability is intrinsically more challenging than the variability of other atmospheric variables such as temperature or geopotential height, which have longer correlation length scales. Instead, rainfall is fundamentally a product of ascent, the residual of the difference in much higher-magnitude wind fields. Thus, reproducing its distribution is a major computational challenge. In an oceanic context, the distribution of SST dictates the general distribution of rainfall, but atmospheric dynamics can substantially alter the distribution of rainfall. Over land, patterns of rainfall are extremely heterogeneous...

Also a stochastic process whose extremes are of extreme human impact. Constraints on extremes are key, but difficult to achieve from the observational record. Finally, the causality of rainfall events can be difficult to ascertain because convection supplies considerable latent heating to the middle troposphere, and this forcing in turn can alter larger-scale circulation fields.

In the Asian monsoon, the difficulties of predicting the distribution of rainfall take on new dimensions given the complexity of landscapes. Mean rainfall rates change by an order of magnitude over just 100 km. It is clear that the distribution of low orography dictates 

Generations of climatologists have attempted to diagnose rainfall from the Asian monsoon, with limited success. Modern efforts such as those released by Skymet or the like are continuously reupdated. Such prediction failures entail massive human imapcts.


\subsection{Space}
The distribution of rainfall with space depends strongly on both time-variant and invariant properties of the atmosphere and land surface: Land-sea contrast, the distribution of orography and sea surface temperature (SST).

\section{Constraints on monthly rainfall rates}
Above all else, this is a study about rainfall, and in particular its heterogeneity, both spatial and temporal. As mentioned in \cite{Trenberth}, the size and intensity of storms is fundamentally constrained by the available supply of water vapor.

\subsection{A Summary of Modern Dynamics}

\section{Tools: Rain Gauge Precipitation}

The satellite record of rainfall is already reasonable with the existence of satellites such as TRMM and the newly-launched GPM, but for obtaining a decadal signal, the only tool available is rain gauge records. They are intrinsically full of potential errors, but thankfully helpful databases exist which compile station records and attempt to remove blatant errors. This thesis is possible in particular thanks to the effort put into developing APHRODITE. APHRODITE suffers from uneven station coverage in space and time, but we have attempted to employ statistical techniques in the work below capable of handling such gappy time series. On a global scale, rain gauge records have been used to deduce the leading mode of variability, which is ENSO-induced.

\section{Research Questions}

\subsection{What Dynamics control the interannual variation of rainfall?}
	
	At the global scale, the leading modes of interannual rainfall variability are better studied. The El Ni\~no-Southern Oscillation (ENSO) is globally the mode of rainfall variability with the largest amplitude. ENSO results from a resonance of equatorial waves made possible by the width of the Pacific Ocean (the narrower Atlantic Ocean does not feature an analogous mode of variability). \cite{Dai} claimed that a global warming mode represented EOF2 of rainfall, featuring XXX and YYY. 	
	
	In the Indian monsoon and East Asian monsoon regions, extensive studies have considered local variability. In the Indian Ocean, the Indian Ocean Dipole (IOD), discovered by \cite{Goswami}, is advanced as the leading pattern of variability. However, a recent study calls into question the existence of the IOD (Sumant Nigam's student).
 Pacific-Japan pattern in East Asia.



\section{Thesis Structure}

Chapter 2 presents an analysis of observational evidence from rain gauge precipitation data suggesting that a link exists between interannual variability in the Indian and East Asian monsoons, and studies the propagation of rainfall anomalies on a daily scale to see if this reproduces leading modes of monthly variability. Chapter 3 presents a hypothesis for a connection between the variability of the Indian and East Asian monsoons based on the findings of the previous chapter, and finds support in dynamical theory and preliminary model data. These results were previously presented in \citep{Day2015}.

Chapter 3 turns to the ``South Flood-North Drought,'' an observed pattern wherein the Yangtze River of China has experienced increased flooding and Northern China severe droughts. We create a new image processing algorithm capable of characterizing daily rainfall data in terms of banded rainfall features (rainbands), and use our technique to describe the South Flood-North Drought in an unprecedented fashion. We also propose a new theory involving the seasonality of the tropospheric jet and its potential ability to induce rainfall anomalies. Chapter 5 returns to the coupling mechanism described in Chapter 3 and tests it in a collection of model simulations using the LMDZ global climate model (GCM) with perturbed boundary conditions. 

In conclusion, Chapter 4 establishes how the information discovered in the course of this thesis may be used on the 
 . essential problem of projecting \nth{21}-century rainfall change due to global warming.