\chapter{Introduction}

\section{Monsoons}

\section{A characterization}

A monsoonal climatology is typified by the summer arrival of consistent, heavy rainfall, and relatively dry conditions during the rest of the year. SHANTARAM QUOTE?!. With its yearly arrival have followed centuries of attempts to prognosticate its advance and severity. Its outcome is linked to rate changes in Indian Gross Domestic Production (GDP) \citep{Gadgil}

The Asian monsoon is dominated by the presence of the Tibetan Plateau, the largest topographic barrier on earth. Standing at over 5,000 meters, the Tibetan Plateau has been the dominant geological feature on Earth for the past 10 million years, and possibly as many as 50 million years. The region therefore functions as a sort of fluid dynamical laboratory whose properties are still not well understood. The earliest modeling studies immediately began to point to the Tibetan Plateau as a potential dominant control of monsoon strength \cite{Manabe}

The underlying motivation for the study of rainfall variability lies in the magnitude of human impacts. On an hourly to daily time scale, extreme rainfall events inflict massive damage and suffering on the Asian Monsoon region. This is borne out culturally by traditions such as the Thai river festival and even the history of Chinese science, wherein sophisticated engineering techniques were developed early in Chinese history for attempting to control the Yangtze River, and culminating in the creation of the Three Gorges Dam. On a monthly time scale, rainfall supply dictates agricultural yields and even long-term demography. In considering the changes in rainfall due to global warming, both shifts in mean and in the tails of the distribution need to be characterized \cite{Pendergrass}.


\section{Rainfall variability}

\subsection{Statistical Properties and Heterogeneity}

The study of rainfall variability is intrinsically more challenging than the variability of other atmospheric variables such as temperature or geopotential height, which have longer correlation length scales. Instead, rainfall is fundamentally a product of ascent, the residual of the difference in much higher-magnitude wind fields. Thus, reproducing its distribution is a major computational challenge. In an oceanic context, the distribution of SST dictates the general distribution of rainfall, but atmospheric dynamics can substantially alter the distribution of rainfall. Over land, patterns of rainfall are extremely heterogeneous...

Also a stochastic process whose extremes are of extreme human impact. Constraints on extremes are key, but difficult to achieve from the observational record. Finally, the causality of rainfall events can be difficult to ascertain because convection supplies considerable latent heating to the middle troposphere, and this forcing in turn can alter larger-scale circulation fields.

In the Asian monsoon, the difficulties of predicting the distribution of rainfall take on new dimensions given the complexity of landscapes. Mean rainfall rates change by an order of magnitude over just 100 km. It is clear that the distribution of low orography dictates 

Generations of climatologists have attempted to diagnose rainfall from the Asian monsoon, with limited success. Modern efforts such as those released by Skymet or the like are continuously reupdated. Such prediction failures entail massive human imapcts.


\subsection{Space}
The distribution of rainfall with space depends strongly on both time-variant and invariant properties of the atmosphere and land surface: Land-sea contrast, the distribution of orography and sea surface temperature (SST).

\section{Constraints on monthly rainfall rates}
Above all else, this is a study about rainfall, and in particular its heterogeneity, both spatial and temporal. As mentioned in \cite{Trenberth}, the size and intensity of storms is fundamentally constrained by the available supply of water vapor.

\subsection{A Summary of Modern Dynamics}

\section{Tools: Rain Gauge Precipitation}

The satellite record of rainfall is already reasonable with the existence of satellites such as TRMM and the newly-launched GPM, but for obtaining a decadal signal, the only tool available is rain gauge records. They are intrinsically full of potential errors, but thankfully helpful databases exist which compile station records and attempt to remove blatant errors. This thesis is possible in particular thanks to the effort put into developing APHRODITE. APHRODITE suffers from uneven station coverage in space and time, but we have attempted to employ statistical techniques in the work below capable of handling such gappy time series. On a global scale, rain gauge records have been used to deduce the leading mode of variability, which is ENSO-induced.

\section{Research Questions}

\subsection{What Dynamics control the interannual variation of rainfall?}
	
	At the global scale, the leading modes of interannual rainfall variability are better studied. The El Ni\~no-Southern Oscillation (ENSO) is globally the mode of rainfall variability with the largest amplitude. ENSO results from a resonance of equatorial waves made possible by the width of the Pacific Ocean (the narrower Atlantic Ocean does not feature an analogous mode of variability). \cite{Dai} claimed that a global warming mode represented EOF2 of rainfall, featuring XXX and YYY. 	
	
	In the Indian monsoon and East Asian monsoon regions, extensive studies have considered local variability. In the Indian Ocean, the Indian Ocean Dipole (IOD), discovered by \cite{Goswami}, is advanced as the leading pattern of variability. However, a recent study calls into question the existence of the IOD (Sumant Nigam's student).
 Pacific-Japan pattern in East Asia.



\section{Thesis Structure}

Chapter 2 presents an analysis of observational evidence from rain gauge precipitation data suggesting that a link exists between interannual variability in the Indian and East Asian monsoons, and studies the propagation of rainfall anomalies on a daily scale to see if this reproduces leading modes of monthly variability. Chapter 3 presents a hypothesis for a connection between the variability of the Indian and East Asian monsoons based on the findings of the previous chapter, and finds support in dynamical theory and preliminary model data. Chapter 4 turns to the ``South Flood-North Drought,'' an observed pattern wherein the Yangtze River of China has experienced increased flooding and Northern China severe droughts. We create a new image processing algorithm capable of characterizing daily rainfall data in terms of banded rainfall features (rainbands), and use our technique to describe the South Flood-North Drought in an unprecedented fashion. We also propose a new theory involving the seasonality of the tropospheric jet and its potential ability to induce rainfall anomalies. Chapter 5 returns to the coupling mechanism described in Chapter 3 and tests it in a collection of model simulations using the LMDZ global climate model (GCM) with perturbed boundary conditions. In conclusion, Chapter 6 considers how the information discovered in the course of this thesis may be used on the essential problem of projecting \nth{21}-century rainfall change due to global warming.