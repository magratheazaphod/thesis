\chapter{Introduction}

\blockquote{
%
%	``On the day the monsoon began, I was swimming in the river with a dozen other young men and about twenty children. The dark clouds, which had painted their sombre moods on the sky for weeks, gathered from horizon to horizon, and seemed to press upon the tops of the tallest trees. The air, after eight dry months, was so lavishly perfumed with rain that we were almost drunk with excitement.
%	
%	`Paous alla! S'alla ghurree!' the children cried repeatedly, grasping my hands. They pointed to the clouds and dragged me toward the village. `The rain is coming! Let's go home!'
%	
%	The first drops of rain fell as we ran. In seconds, the drops were a heavy fall. In minutes, the fall was a cascade. Within an hour, the monsoon was a ceaseless torrent, so thick that it was difficult to breathe in the open without cupping my hands to my mouth to make a little cave of air.
%	
%	At first, the villagers danced in the rain and played pranks on one another. Some took soap, and washed in the heaven-sent shower. Some went to the local temple, where they knelt in the rain to pray. Others busied themselves with repairs to the roofs of their houses and the drainage trenches dug around every mud-brick wall.
%	
%	Eventually, everyone stopped to simply stare at the drifting, flapping, curling sheets of rain. Every doorway of every house was crowded with faces, and each flash of lightning showed the frozen tableaux of wonder.
%	
%	That downpour of several hours was followed by a lull just as long. The sun shone intermittently, and rainwater steamed from the warming earth. The first ten days of the season proceeded in the same way, with violent storms and tranquil lulls, as if the monsoon was probing the village for its weaknesses before mounting a final assault.
%	
%	Then, when the great rain came, it was a lake of water in the air, and it rained almost without pause for seven days and nights. On the seventh day, I was at the river's edge, washing my few clothes as the drenching torrents fell. At one point I reached for my soap, and realised that the rock I'd place it on was submerged. The water, which had merely caressed my bare feet, rose from my ankles to my knees in seconds. As I looked upstream at the tumbling crash of the river, the water reached to my thighs, and was still rising.
	
	`The river! The river is coming!' I shouted, in broken Marathi.

	Sensing my distress but not really understanding me, the villagers gathered around and then called Prabaker, plying him with questions.
	
	`What is your matter, Lin? The people are very upset for you.'
	
	`The river! It's coming up fast. It'll wipe the village out!'
	
	Prabaker smiled.
	
	`Oh, no, Lin. That will not be happening.'
	
	`I'm telling you! I've seen it. I'm not joking, Prabu. The fucking river's in flood!'
	
	Prabaker translated my words for the others. Everyone laughed.
	
%	`Are you all crazy?' I shouted, in exasperation. `It's not funny!'
%	
%	They laughed all the harder and crowded around me, reaching out to calm my fear by patting and stroking me, their laughing voices full of soothing words and sighs. Then, with Prabaker leading the way, the crowd of villagers goaded, dragged and pushed me toward the river.
	
	The river, only a few hundred meters away, was a deluge: a vast muddy concrescence that tore through the valley in heaving waves and boiling eddies. The rain redoubled its intensity as we stood there, our clothes as drenched as the yielding soil. And still the tumid river grew, consuming new land with each thumping heartbeat.
	
	`You see those sticks, Lin,' Prabaker said, in his most irritating attempt at a soothing tone. `Those sticks are the flood-game sticks. Do you remember, when the people put them in the ground? Satish and Pandey, Narayan and Bharat...do you remember?'
	
%	I did remember. Days before, there'd been a lottery of some kind ... 
%	
%	One hundred and twelve numbers - one for every man in the village - were written on small pieces of paper, and mixed together in an empty clay water-pot, called a matka. The men lined up to draw their numbers, and then a second set of the same numbers was mixed in the pot. A little girl was given the honour of drawing the six winning numbers from the pot. The whole village watched the ceremony, and applauded the winners happily.
%	
%	The six men whose numbers had been drawn had won the chance to hammer a wooden stake, a little over a metre long, into the earth. As well, the three oldest men in the village were accorded the right to a wooden stake without the numbered lottery. They duly chose places for their stakes, and younger men obliged by hammering the wooden pegs into the ground. When all nine stakes were positioned, little flags with the names of the men were tied to each one, and the people drifted back to their homes.
%	
%	I'd watched the affair from a shady spot beneath the branched dome of a tree. At the time, I was working on my own small reference dictionary of the Marathi language, based on phonetic spellings of the words I heard every day in the village. I gave the ceremony little attention, and I never bothered to ask its purpose.
	
	I did remember. Days before, there'd been a lottery of some kind ... As we stood in the numbing, drumming rain and watched the prowling advance of the river, Prabaker explained that the wooden stakes were part of a flood-game that was played every year. The oldest men in the village, and six lottery winners, were given the chance to predict the point to which the river would rise. Each wooden stick, with its flag of yellow silk, represented a best guess.
	
	`You see, this one little flag?' Prabaker asked, pointing to the stake that was furthest from where we stood. `This one is almost gone. The river will reach to him, and cover him, tomorrow or tonight.' 
	
	He translated what he'd told me for the crowd, and they pushed Satish, a heavy-set cowherd, to the front of the group. The almost submerged stick was his, and he accepted, with shy laughter and downcast eyes, the good-natured jeers of his friends and the sneers of the older men.
	
	`And this one here,' Prabaker went on, pointing to the stake nearest to our position. `This one is the river will never be touching. The river never comes more far than this place. Old Deepakbhai has picked for himself this place, for the putting of his stick. He thinks this year will be a very heavy monsoon.'
	
%	The villagers had lost interest, and were already drifting or jogging back to the village. Prabaker and I stood alone.
	
	`But ... how do you know that the river won't rise past this point?'
	
	`We are here a long time, Lin. Sunder village has been in this place for two thousands of years. The next village, Natinkerra, has been there for much longer, about three thousands of years. In some other places - not near to here - the people do have a bad experiences, with the floods, in monsoon time. But not here. Not in Sunder. Our river has never come to this far. This year, also, I don't think it will come to this far, even so old Deepakbhai says it will. Everybody knows where the river will stop, Lin.' \attrib{Gregory David Roberts, \textit{Shantaram}, 133-135}}
	
	Clearly, atmospheric dynamicists are not the only ones to engage in the art of monsoon prognostication. Nonetheless, in spite of a century of work, forecasts of present-day monsoon seasonal variability and of future changes under global warming remain nebulous. The purpose of this dissertation is to elucidate the atmospheric dynamics implicated with the leading mode of Asian summer monsoon variability and reveal connections to other important modes of climate variability. The ultimate vision is that increasingly skillful projections of these other climate components will also be used to improve projection of changes in the \nth{21} century monsoon.

	An overview of the current consensus on monsoon dynamics and other information on the spatiotemporal variability of rainfall in the Asian monsoon are presented in the introduction to Chapter 2. Instead, I include here short overviews of several topics that are not addressed in any of the following chapters: the history of the monsoon on paleoclimatic timescales, human impacts of monsoon variability, and the consensus of current projections on the future of the monsoon. These are not intended to be exhaustive reviews, but rather background information that justifies the importance of the material in following chapters.
	
\section{The Asian Paleomonsoon}

	The study of rainfall variability in the Asian monsoon suffers from the limited duration of available records. Satellite observations provide an even spacial distribution of rainfall as well as instantaneous snapshots of the same region at different points in its diurnal cycle. The satellite record has continued to expand thanks to the unexpected longevity of the Tropical Rainfall Measuring Mission (TRMM) and the relatively recent Global Precipitation Measurement (GPM) satellite, but unfortunately extends back only to 1997. Instead, we rely on rain gauge data for decadal time series, assembled from daily collection of precipitation at weather stations around the globe, often with very basic apparatus. Unlike satellite observations, records are available only over land and are highly heterogeneous with space and time, and also subject to observational errors. Data sets such as APHRODITE, a rain gauge data set focused on the Asian monsoon region, attempt to weed oust spurious observations with quality control algorithms and then present users with a refined product \citep{Yatagai2012}. This dissertation could not have been written without the efforts of APHRODITE's compilers; they are further acknowledged in subsequent chapters.
	
	We would also like to consider how the monsoon operates under substantially altered climates. The planet is on the verge of reaching a global mean temperature unseen in recent Earth history, and it could be useful to have information on how the monsoon behaved under altered insolation conditions, and how much it can be perturbed from its present form in general. Fortunately, remarkable records exist that can provide information about rainfall on a precessional time scale: cave speleothems, which are stalagmites featuring tens of thousands of years of continuous deposition\footnote{At some sites such as in Oman, stalagmite deposition is intermittent, also taken as an indicator of paleomonsoon change \citep{Burns2001,Fleitmann2003}.} from which a $\delta ^{18}O$ time series can then be compiled, often by stitching together records from multiple stalagmites. Since 2001, dozens of speleothem sites have been developed across East and South Asia, particularly in the East Asian monsoon region \citep{Wang2001,Dykoski2005,Wang2008b}, but also near the Tibetan Plateau \citep{Cai2010b}, Yunnan Plateau \citep{Cai2015}, Altay Mountains\citep{Cheng2012} and numerous other locations. There are fewer of such records in India, partially because caves are often holy sites in local folklore.
	
	Beyond the difficulty in obtaining such records, their interpretation poses a major challenge. East Asian speleothem $\delta ^{18}O$ records tend to feature remarkable agreement across thousands of kilometers, most notably demonstrating simultaneous abrupt transitions on the time scale of precessional forcing \citep{Chiang2015}. Past authors have interpreted these coherent changes as variations in Asian monsoon ``intensity'', and therefore local rainfall \citep{Wang2001,Liu2014}, based on the concept of the ``amount effect'' wherein the $\delta ^{18}O$ of precipitation from a parcel will decrease over time as heavier isotopes rain out, a process referred to as Rayleigh distillation \citep{Dansgaard1964}. However, in present day observations, the climatological distribution of precipitation $\delta ^{18}O$ is poorly explained by Rayleigh distillation, and furthermore monthly variations in precipitation $\delta ^{18}O$ across China are poorly correlated with local precipitation and temperature \citep{Dayem2010,Lee2012}. Instead, China $\delta ^{18}O$ is weakly correlated with Indian monsoon rainfall amounts upstream \citep{Lee2012}. Therefore, recent interpretations have turned to alternative mechanisms, such as changes in moisture source region, variations in Indian monsoon rainfall intensity and changes in transport pathway \citep{Maher2008,Dayem2010,Pausata2011,Baker2015}. That being said, no single theory has explained the abrupt jumps in East Asian speleothem records in a fully satisfactory way, or proven whether they correspond to abrupt changes in Asian monsoon rainfall or not.
	
	On a time scale of millions to tens of millions of years, the Asian monsoon region can be viewed as a geophysical fluid dynamics laboratory. The Tibetan Plateau rose to its full height of 5,000+ meters some time in the past 50 Ma after the collision of the Indian sub-continent, with massive climate impacts. Based on geological evidence, it was previously proposed that a surge in Tibetan Plateau elevation around \mytilde10 Ma may have triggered a sudden intensification of the Indian monsoon \citep{Harrison1992,Molnar1993}. However, recent work suggests that a sudden jump in preserved \textit{G. bulloides} deposits in the Arabian Sea around 8.5 Ma, previously attributed to monsoon intensification, likely resulted instead from uplift of the Indian Ocean seafloor \citep{Rodriguez2014}. In addition, theoretical studies suggest that a strong monsoon should exist as long as there remains a region of strong off-equatorial heating, provided by the presence of the Indian subcontinent since 50 Ma \citep{Prive2007a,Bordoni2008,Molnar2010}. 
	
	One other intriguing multi-million year record exists: northern China's Loess Plateau, the longest continuous depositional basin on the planet. By preserving millions of years of silt and dust from surrounding deserts and the rest of the Northern Hemisphere, the Loess Plateau stores invaluable information about paleoclimate and in particular glacial-interglacial cycles, since the planet tends to be dustier during glacials \citep{Sun2006}. However, like the speleothem records, its interpretation remains a point of contention \citep{Roe2009}.

\section{Human Impacts}

	In summer, monsoonal regions experience bouts of steady, heavy rainfall interspersed by occasional ``breaks,'' and much drier conditions during the rest of the year. In many parts of Asia, this yearly supply is used to grow two crops per year; in India, these two crops are the summer \textit{kharif} and the winter \textit{rabi}, both of whose fruition depends on rainfall supply from the monsoon \citep{Gadgil2006a}. The majority of Indian land in the 2000s remained unirrigated and depended directly on rainfall \citep{KrishnaKumar2004}. Therefore, the timing and intensity of the summer monsoon control agricultural outcomes. An early or delayed Indian monsoon onset can interfere with the transplantation of rice seedlings from their nursery beds \citep{Gadgil2006a}. Most Indian crop yields are correlated with June-September All-India monsoon rainfall above a 99\% confidence level \citep{KrishnaKumar2004}, and severe drought All-India Monsoon Rainfall years were found to induce cuts to Indian GDP of 2-5\% \citep{Gadgil2006}. Aside from agriculture, extreme rainfall events can inflict massive damages through flooding \citep{Li2012}. The research literature estimating damages from monsoon-season flooding is sparse, although studies of cyclones find that long-term mortality can exceed immediate mortality by an order of magnitude \citep{Anttila-Hughes2012}, and that countries experiencing \nth{90} percentile cyclone events  suffer from a per-capita income reduction of 7.4\% 20 years later versus a scenario where the event did not happen \citep{Hsiang2014}.
	
	It can be challenging to interpret the historical record of the past fifty years because humans have induced not only warming due to increasing CO$_2$ but also massive anthropogenic forcing from emissions of NO$_x$, black carbon and other aerosols. It is estimated that high levels of near-surface ozone due to human activity lower India's crop yields by 9\% each year, enough to feed roughly 94 million people \citep{Ghude2014}, and that all noxious emissions combined induce a 36\% decline in wheat yields \citep{Burney2014}. The massive injection of aerosols to East Asia has cut down on the amount of insolation directly reaching the surface by (so-called ``solar dimming'') when the rest of the planet has experienced a solar ``brightening'' \citep{Norris2009}. In China, the changes in climate observed over the past half-century have been blamed on both aerosols and warming with no clear consensus \citep{Menon2002,Yang2015,Yu2016,Yang2016}.
	
\section{\nth{21} Century Projections}

	Changes in rainfall under global warming have been extensively studied. The principal finding across model simulations has been that rainfall rates will increase at a slower rate than the rate of increase of global surface temperature \citep{Allen2002}. In considering the changes in rainfall due to global warming, both shifts in mean and in the tails of the distribution need to be characterized \cite{Pendergrass}. As mentioned in \citet{Trenberth}, the size and intensity of storms is fundamentally constrained by the available supply of water vapor. Constraints on extremes are key, but difficult to achieve from the observational record.
	

	In the Asian monsoon, the difficulties of predicting the distribution of rainfall take on new dimensions given the complexity of landscapes. Mean rainfall rates change by an order of magnitude over just 100 km. It is clear that the distribution of low orography dictates 

	Generations of climatologists have attempted to diagnose rainfall from the Asian monsoon, with limited success. Modern efforts such as those released by Skymet or the like are continuously reupdated. Such prediction failures entail massive human impacts.
	
	CITE INNOVATIVE APPROACHES SUCH AS JINQIANG'S THESIS, BOOS LINEAR MONSOON PAPER, ETC
	
	Merging the themes of the previous two sections, many have investigated the human impacts of further warming. Rising nighttime temperature has been indicated as a threat to agricultural productivity, decreasing yields by as much as 10\%/$^{\circ}$K \citep{Peng2004}. Many crops are found to have a critical temperature, wherein yields will rise steadily up until that temperature and then decline severely if it is exceeded \citep{Schlenker2009}. In Indonesia, global warming in the IPCC4 model suite was suggested to raise the probability of a 30-day delay in monsoon initiation from 9-18\% to 30-40\% by 2050 \citep{Naylor2007}. \citet{Auffhammer2012} found that agricultural productivity in India was 4\% lower during 1966-2002 had monsoon conditions from 1960 persisted through that era. A new literature focuses on linking climate more directly to human impacts. For instance, \citet{Burke2015} showed that economic productivity is nonlinear with temperature across the globe, with peak productivity occurring at 13$^{\circ}$C and decreasing nonlinearly at higher temperature.  Conditions are ripe for the application of such methods to the South Asian and East Asian monsoon regions, but to our knowledge no efforts to date have focused specifically on those regions.

\section{Thesis Structure}

	Chapter 2 presents an analysis of observational evidence from rain gauge precipitation data suggesting that a link exists between interannual variability in the South Asian and East Asian monsoons, and studies the propagation of rainfall anomalies on a daily scale to see if this reproduces leading modes of monthly variability. We propose that the variability of the two monsoons is linked by monthly variations in moisture transport from the Bay of Bengal to the Yangtze River valley across the Yunnan Plateau, a spur of terrain to the southeast of the Tibetan Plateau. Dynamical theory and preliminary model data support major elements of this theory. These results were previously presented in \citep{Day2015}.

	Chapter 3 develops the Rainband Detection Algorithm (RDA), a recursive image processing tool to analyze daily patterns of frontal rainfall in China. We discover that banded rainfall contribute a large fraction of yearly total precipitation to Central China and describe the seasonal progression of banded rainfall. Our algorithm also allows a novel characterization of the South Flood-North Drought, a known pattern of decadal change wherein central China has experienced increased rainfall and northern China severe droughts beginning in the 1980s. Most notably, rainbands have become \textit{less frequent} in May and \textit{shifted southward} in July-August. Finally, we show that the pattern of decadal change in yearly rainfall totals is due specifically to change in banded rainfall.

	Lastly, Chapter 4 returns to the leading mode of Asian summer rainfall variability and shows robust links with two other important climatic components: The Circumglobal Teleconnection, a high-wavenumber Northern Hemisphere standing wave responsible for heat waves, and the East Asian tropospheric jet. Using JRA-55 reanalysis data, we find that changes in westerly moisture transport across the Yunnan Plateau drive a zonal band of heating spanning the Himalayan Foothills and Yangtze River valley, which we propose can then stimulate the CGT mode and contributes to its phase-locking. Also associated with these changes are meridional shifts of the East Asian jet. Using the RDA catalog from Chapter 3, we are able to show that the two major changes in rainbands associated with the South Flood-North Drought also corresponded to southward shifts of the East Asian jet. We suggest that these findings have the potential to improve projection of the \nth{21}-century Asian monsoon. 

\section{Future directions}

	The study of rainfall variability is intrinsically more challenging than the variability of other atmospheric variables such as temperature or geopotential height, which have longer correlation length scales. Instead, rainfall is fundamentally a product of ascent, the residual of the difference in much higher-magnitude wind fields. Thus, reproducing its distribution is a major computational challenge. Therefore, the advent of high resolution global climate modeling will change the study of the Asian monsoon. Existing studies already show that low, narrow orographic barriers greatly impact the monsoon's climatology \citep{Xie2006}. Studies with high resolution to date already show that resolving features that used to be sub-grid scale changes the hydrologic balance and circulation \citep{Risi2010,Boos2013a,Wu2014a,Wu2016}. Such modeling will allow us to test the central questions broached in the following chapters. In particular the following questions should be answered: if moisture transport across the Yunnan Plateau couples the Indian and East Asian monsoons, what would happen with a Tibetan Plateau-height Yunnan Plateau? What about in the absence of the Yunnan Plateau?
	
	Above all else, this is a study about rainfall, and in particular its heterogeneity, both spatial and temporal. 
	
	The holy grail remains a robust projection for what will happen to the Asian monsoon under further warming. Under the simplified template of the monsoon as an oversized sea-breeze, more heating of land, and also higher insolation, were associated directly with greater rainfall. The idealized modeling studies 