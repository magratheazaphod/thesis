\chapter{Introduction}

\section{Constraints on monthly rainfall rates}
Above all else, this is a study about rainfall, and in particular its heterogeneity, both spatial and temporal. As mentioned in \cite{Trenberth}, the size and intensity of storms is fundamentally constrained by the supply of 

\section{Heterogeneity}
Distributions of rainfall vary on shorter distances than other atmospheric variables.

\subsection{Space}
The distribution of rainfall with space depends strongly on both time-variant and invariant properties of the atmosphere and land surface: Land-sea contrast, the distribution of orography and sea surface temperature (SST).

\subsection{Time}
\section{Human Impacts of Rainfall Extremes}
\section{Monsoons}
\subsection{A Summary of Modern Dynamics}

\section{Tools: Rain Gauge Precipitation}

\section{Research Questions}

\subsection{What Dynamics control the interannual variation of rainfall?}

\section{Thesis Structure}

Chapter 2 presents an analysis of observational evidence from rain gauge precipitation data suggesting that a link exists between interannual variability in the Indian and East Asian monsoons, and studies the propagation of rainfall anomalies on a daily scale to see if this reproduces leading modes of monthly variability. Chapter 3 presents a hypothesis for a connection between the variability of the Indian and East Asian monsoons based on the findings of the previous chapter, and finds support in dynamical theory and preliminary model data. Chapter 4 turns to the ``South Flood-North Drought,'' an observed pattern wherein the Yangtze River of China has experienced increased flooding and Northern China severe droughts. We create a new image processing algorithm capable of characterizing daily rainfall data in terms of banded rainfall features (rainbands), and use our technique to describe the South Flood-North Drought in an unprecedented fashion. We also propose a new theory involving the seasonality of the tropospheric jet and its potential ability to induce rainfall anomalies. Chapter 5 returns to the coupling mechanism described in Chapter 3 and tests it in a collection of model simulations using the LMDZ global climate model (GCM) with perturbed boundary conditions. In conclusion, Chapter 6 considers how the information discovered in the course of this thesis may be used on the essential problem of projecting \nth{21}-century rainfall change due to global warming.