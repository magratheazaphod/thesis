\chapter{Coupling of South and East Asian monsoon precipitation in July-August}

\section{Abstract}

The concept of the ``Asian monsoon'' masks the existence of two separate summer rainfall r\'egimes: Convective storms over India, Bangladesh and Nepal (the South Asian monsoon), and frontal rainfall over China, Japan and Korea (the East Asian monsoon). In addition, the Himalayas and other orography, including the Arakan Mountains, Ghats and Yunnan Plateau, create smaller precipitation domains with abrupt boundaries. We find a mode of continental precipitation variability that spans both South and East Asia during July and August. Point-to-point correlations and EOF analysis with APHRODITE, a 57-year rain gauge record, show that a dipole between the Himalayan Foothills (+) and the ``Monsoon Zone'' (Central India, -) dominates July-August interannual variability in South Asia, and is also associated in East Asia with a tripole between the Yangtze Corridor (+) and North and South China (-). July-August storm tracks, as shown by lag-lead correlation of rainfall, remain mostly constant between years and do not explain this mode. Instead, we propose that interannual change in the strength of moisture transport from the Bay of Bengal to the Yangtze Corridor across the northern Yunnan Plateau induces widespread precipitation anomalies. Abundant moisture transport along this route requires both cyclonic monsoon circulation over India and a sufficiently warm Bay of Bengal, which coincide only in July and August. Preliminary results from the LMDZ5 model, run with a zoomed grid over Asia and circulation nudged to ECMWF reanalysis, support this hypothesis. Improved understanding of this coupling may help to project \nth{21} century precipitation changes in East and South Asia, home to over 3 billion people.

\section{Previously known modes of variability}

\subsection{South Asia}
Monsoon Intraseasonal Oscillations and the Madden-Julian Oscillation.

\subsection{East Asia}

\section{Point-to-Point Correlations}

\subsection{An Agreement Map Methodology}

\section{Empirical Orthogonal Function Analysis}