\chapter{Propagation of Asian storms: Lag-lead Correlation of Rainfall Anomalies}

In Chapter 2, we found that July-August rainfall anomalies are linked between the South Asian and East Asian monsoons. In this section, we test a simple first hypothesis: That the spatial pattern of correlation reflects either a change in frequency along an existing storm track, or a shift in storm tracks.

\section{Previous Work} %must expand considerably for thesis.
	
	In the search for a process that connects South and East Asia, we investigate the propagation of storms. A simple first hypothesis is that the patterns observed in Figure 4 and All-Asia JA spatial EOF1 correspond to interannual changes in the frequency or trajectory of storms. Storms in the Asian monsoon can propagate across thousands of kilometers, but interface with topography in complex ways \citep{Romatschke2011a}. \cite{Luo2011} used CloudSat and CALIPSO satellite data to determine the horizontal and vertical length scales of storms in different regions (India, the Tibetan Plateau and East Asia), and found that storms on the Tibetan Plateau are shallower and have a shorter horizontal length scale than storms in India. One possible interpretation of this result is that storms do not cross between India and the Tibetan Plateau. However, it has been known for decades that vortices on the Tibetan Plateau may, depending on synoptic conditions, propagate downstream to eastern China, where they induce heavy rainfall and potential flooding \citep{Tao1981,Murakami1984,Chen1984,Yasunari2006,Xu2011,Wang2012a}. Likewise, depressions from west Pacific tropical cyclones can cross Indochina and reach India from July to September depending on background circulation \citep{Chen1999,Fudeyasu2006}. Thus, storms can propagate between South and East Asia under some circumstances. We quantify this behavior below.
	
\section{Formula}

	Past studies have used HYSPLIT (Hybrid Single Particle Lagrangian Integrated Trajectory) analysis to create back-trajectories of air parcels in Asia during monsoon season \citep{Medina2010,Cai2012,Gao2013}. However, HYSPLIT uses circulation obtained from reanalysis products, which struggle to produce realistic frequency distributions of precipitation in the region \citep{Pena-Arancibia2013}. As an alternative, we use lag-lead correlation with APHRODITE to extract the propagation of precipitation anomalies across days. This analysis cannot show all storms, for instance storms that do not produce rainfall or that do not propagate across multiple days, but suffices to study the passage of storms between South and East Asia, which is our focus.
	
	For a reference point $i$ with normalized anomaly time series $P''_i$ and a phase lag of $\lambda$ days, the lag-lead correlation $c_i^\lambda(x,y,yr)$ with rainfall at another point $x,y$ is given by:
\begin{gather*}
	c_i^\lambda(x,y,yr)=\sum_{days}P''_i(day,yr)*P''(x,y,day+\lambda,yr),\\
	\text{for } \lambda \text{= -5 to +5 days and year = 1951 to 2007}
\end{gather*}
	
	 This is identical to the formula for the correlation coefficient $r$ with an offset of $\lambda$ days between time series (a lag or lead depending on the sign of $\lambda$). APHRODITE cannot provide information on sub-daily variation, propagation over oceans, or different mechanisms of propagation. However, the 57 years of data can be used to extract both mean storm trajectories and their interannual variability. The $c_i^\lambda$ require further processing to isolate propagation because there tends to be a nonzero positive background field \textit{independent of the value of} $\lambda$. This background field, different for each reference point $i$, results from several effects, including the false positive correlation of two points without rain, even if they are distant from one another, and also the deviation of precipitation anomalies from a normal distribution. We define the background field $b_i(x,y)$ as the mean lag-lead correlation averaged over all $\lambda$ and years, and thereafter analyze the \textit{anomaly} from this background, $C_i^\lambda(x,y,yr)$, and the 57-year mean anomaly $K_i^\lambda(x,y)$:
\begin{align*}
	b_i(x,y) &=\overline{c_i^\lambda(x,y,yr)}^{57\text{ years, }\lambda = \text{-5 to +5}} \\
	C_i^\lambda(x,y,yr) &= c_i^\lambda(x,y,yr)-b_i(x,y) \\
	K_i^\lambda(x,y) &= \overline{C_i^\lambda(x,y,yr)}^{57\text{ years}}
\end{align*}
		
	We calculate $C_i^\lambda(x,y,yr)$ and $K_i^\lambda(x,y)$ at every reference point for $\lambda =$ -5 to 5 and from 1951 to 2007. Figure 8 shows $K_i^\lambda$ for reference points $i=$ 2, 6, 13, 16 and 21 (Kathmandu, Durg, Shenzhen, Enshi and Baotou) as well as two additional sites, Lijiang (100.4\textdegree E 26.9\textdegree N) and Lake Qinghai (100.1\textdegree E 37.4\textdegree N). In addition, for each reference point and lag $\lambda$, we find the location of maximum $C_i^\lambda(x,y,yr)$ in each of the 57 years, and then draw the smallest circle that contains at least 50\% of each of the yearly maxima. This quantifies interannual variability. Figure 9 condenses propagation information from Figure 8 into a single composite image by showing the lag $\lambda$ for which $K_i^\lambda(x,y)$ is maximized, with 50\% variance circles for selected $\lambda$ and connecting arrows superimposed. Using these tools, we focus on whether storms propagate between South and East Asia, whether storm tracks change between years, and what trajectories reveal about underlying dynamics.
	
\section{Results}	
		 	 		
	 In Figure 8, $K_i^0$ ($\lambda =$ 0) reveals the size of storms at each reference point, typically around 300 km. Interannual variability is generally small for $\lambda$ = -2 to 2. Negative values of $K_i^\lambda(x,y)$ may result from a strong positive $K_i^{\lambda}$  on another day, and should not necessarily be interpreted as storm suppression. All reference points show coherent propagation of anomalies across days. 
	 
	 We focus first on the South Asian monsoon domain. In the ``Monsoon Zone'' (Figure 8b, Durg), storms propagate west-northwestward from the Bay of Bengal with little variance in trajectory, also seen in past work such as Figure 1 of \cite{Sikka1977}. These storms, known in the literature as ``monsoon depressions'' or ``low-pressure systems''\citep{Sikka1977,Chen1999,Krishnamurthy2010}, generally do not reach tropical cyclone intensity. Instead, tropical cyclone occurrence in the Bay of Bengal is confined mostly to October-November and April-May \citep{Li2013}. Several previous studies show that monsoon depressions can originate from further east over Indochina or the South China Sea \citep{Saha1981}. Storms reaching Kathmandu (Figure 8a) also propagate westward, but their primary source is the Yunnan Plateau to the east, with a contribution from Bangladesh and the Bay of Bengal visible at $\lambda =$ -1. In turn, Figure 8d (Lijiang) shows that these Yunnan Plateau storms originate from the mid-latitude westerlies north of the Tibetan Plateau ($\lambda =$ -5 to -2). Bay of Bengal depressions do not reach the Yunnan Plateau. The Himalayas divide regions of westerly and easterly propagation. Figures 8a and 8b also indicate that rainfall peaks over the Himalayan Foothills and South India 5 days before and after a storm passes through the ``Monsoon Zone,'' and vice-versa. This reflects the spatial pattern associated with ''intraseasonal oscillations,'' or ISOs, an extensively studied 10-20 day mode of variability associated with the cycle of active and break periods in the South Asian monsoon \citep{Krishnamurti1980,Chen1993,Annamalai2001,Han2006,Fujinami2011,Fujinami2014}.
	 	 
	 In East Asia, the direction of propagation also shifts from westerly north of 30\textdegree N to easterly over South China. In Figures 8c and 9c (Shenzhen), storms from the Philippines and Taiwan move northwestward to South China and then westward toward the Yunnan Plateau, with low interannual variability for $\lambda =$ -2 to 2. This behavior has been seen both in observation \citep{Chen1999,Liu2003} and in idealized monsoon studies \citep{Prive2007a}. Baotou, our northernmost reference point (Figures 8g and 9g), sits at the July-August latitude of the tropospheric jet \citep{Schiemann2009}, and propagation is therefore strictly westerly. Central China marks the transition between westerly and easterly storm advection. Over Enshi (Figures 8e and 9e, Yangtze Corridor), westerly storms are sheared into northeast-southwest tilted bands. This phenomenon, also seen in Figures 8f and 9f (Lake Qinghai), can be understood by considering upper-level winds at this latitude (Figure 10a). If a midlatitude storm transported by the westerly jet is perturbed southward, it will gain westward velocity from mean flow, whereas storms passing further to the north continue eastward. The Himalayas block the passage of storms between the Tibetan Plateau and India, but storms are able to traverse the lower terrain of the Yunnan Plateau.
	 
\section{Correspondence to Upper Tropospheric Winds}	 
	 
	 In all regions, the direction of propagation agrees closely with 200 mb-level winds (Figure 10a). For the previously discussed ``monsoon depressions,'' this observation is in fact a coincidence. The west-northwestward travel of monsoon depressions instead results from an unusual secondary circulation that advects disturbances against lower-level flow without interacting with the upper troposphere \citep{Chen2000,Chen2005}, even though the predecessors of monsoon depressions are brought by upper-level winds from the east. In the rest of Asia, upper-level winds steer disturbances. We verify this claim by also performing lag-lead correlations for the months of June and September (not shown). Trajectories are mostly similar, and substantial changes correspond to changes in 200 mb-level winds. The low interannual variability of storm trajectories results from the relative constancy of upper-level winds between years, as seen for instance with tropical cyclones in the western Pacific \citep{Kumar2005}. The band of positive correlation in All-Asia JA EOF1 does not correspond to the storm tracks in Figures 8 and 9, nor to their interannual variations. Figures 10c and 10d show the 200-mb level winds associated with the 5 most positive EOF1 years (``wet'' years) and 5 most negative years (``dry'' years). The steering direction of storms remains steady in both, although some changes occur. A check of the $K_i^\lambda$ in these ``wet'' and ``dry'' years also does not reveal major differences (not shown). Therefore, we propose that the interannual variability of storm trajectories does not explain the correlation of precipitation anomalies between South and East Asia.
	 
\section{Storms Fail to Explain the Coupling of South and East Asian monsoon rainfall: An explanation}
	 
	 Storms are the proximate cause of precipitation, and yet Figure 9 shows that July-August storm trajectories behave differently from monthly rainfall anomalies. Both respond to blocking by the Himalayas, but storm trajectories are less responsive to other low topography. The propagation direction of storms is roughly a function of latitude, without the local heterogeneity observed in rainfall. Lastly, the direction of storm tracks does not change much from year to year. Storms produce rainfall and yet appear incapable of explaining its variations on longer time scales.
	 
	 A solution can be identified by considering northeastern India and the southeastern Tibetan Plateau. Although Figure 9d shows that storms in the region come from the Yunnan Plateau to the east, local observations of $\delta^{18}$O show a Bay of Bengal origin and isotopic depletion from convection \citep{Gao2011}. The seeming incompatibility of storms and vapor history helps us to isolate two separate processes: Storm propagation and moisture transport, both of which interact with mean flow in different ways. Storms are an eddy process superimposed on the mean state of the atmosphere. Synoptic depressions are steered by the upper troposphere and recycle whatever water vapor is locally available as they propagate. In contrast, because the scale height of water vapor is about 3 km, moisture transport depends on the state of the lower troposphere, where patterns of convergence change greatly from year to year \citep{Annamalai2001,Yoon2005}. The fixity of storm trajectories points to changes in moisture transport as the root of interannual precipitation anomalies. In the next section, we propose a mechanism whereby such changes may induce coupling between South and East Asia.
