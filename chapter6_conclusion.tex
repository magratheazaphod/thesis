\chapter{Conclusion}

This section pursues two goals: First, to restate the discoveries made in the course of my doctoral research; Second, to outline potential future directions stemming from this work.

\section{My discoveries}

\subsection{Coupled monthly variability of the South Asian and East Asian monsoons in July-August}

\subsection{Coordinated changes between the Meiyu Front and tropospheric jet}

\section{Future Directions}

	A new generation of computer models promises to resolve convection at the ~10-km scale while maintaining global dynamic and energetic consistency. Improved resolution of boundary conditions will lead to realistic rainfall patterns, but if models are not resolving the relevant processes that supply rainfall to a given region, the projections obtained cannot be trusted, especially under \nth{21} century greenhouse gas forcing. My work suggests a pathway wherein we do not simply rely on the power of improved model resolution to constrain projections (indeed IPCC AR5 projections are essentially unchanged from those in IPCC AR4), but try to better understand the physical processes that deliver rainfall to key regions, and the changes in the process themselves. The models can subsequently be tested for their ability to reproduce such key processes. In style, we propose a technique akin to \cite{Boos} and \cite{Tierney}, who evaluated models based on their ability to produce circulation patterns known from paleoproxy data or from atmospheric soundings and reanalysis data.
	
	We have argued in our study that the South Asian monsoon and East Asian monsoon are linked by the transport of moisture across the Yunnan Plateau. Furthermore, this transport is found to be the product of internal variability, and not just a reflection of global changes. A change in the coupling between the two regions has the potential to alter the mean distribution of Asian monsoon rainfall and its overall hydrological budget. We have found in Chapter X that changes in Bay of Bengal SST promote increased moisture transport across the Yunnan Plateau. The anticipated change in the region is ... . Therefore, a first projection would be that transport along this pathway will strengthen/weaken. These changes should be borne out by increased aridity of the eastern Tibetan Plateau and Yunnan Plateau. The observational record on the Tibetan Plateau of the past two decades shows striking changes, featuring ... . It has been suggested that the forcing is anthropogenic. From our study, we project a continued pattern ... blah blah.
	
	The vicinity of the Tibetan Plateau suffers from a limited observational record, but a new generation of measurements hold the prospect of constraining the atmospheric dynamics of the region, and confirm or rejecting existing hypotheses about the dynamic role of the Tibetan Plateau.
 
	In terms of the practical consequences of global warming, changes in mean and extreme rainfall are surely the largest impact. Two immediate consequences include impacts on agriculture, and changes in the frequency of drought and flooding. The principal finding of the previous decade has been that rainfall does not increase with Clausius-Clapeyron scaling (7$\%^{\circ}$K), but at a slower rate of 2-3$\%^{\circ}$K globally. Meanwhile, extreme rainfall events increase at a higher scaling rate, albeit still below Clausius-Clapeyron. Simple scalings have been proposed to explain this global change in rainfall. However, smaller regions can experience much larger changes in rainfall such as a shift in convective margins \cite{Lintner2007}.

	A central finding of this study is that low orography has a major impact on the distribution of mean rainfall. In addition, the local response to broader modes of variability can feature sharp regime transitions. Moist static energy can help to project the statistical mean distribution of rainfall, and may even be correlated with its interannual variability in tropical regions, but shows some weaknesses: ...

	In combination with the trend of desertification of northern China, a decrease in precipitation in northern China has the potential of changing a wide swath of agricultural heartland into inarable terrain. The Chinese government has already resorted to techniques such as the ``New Great Wall,'' a line of trees 200 km westward of Beijing, as well as the South-North Water Diversion Project.