\chapter{Conclusion}

This section pursues two goals: First, to restate the discoveries made in the course of my doctoral research; Second, to outline potential future directions stemming from this work.

\section{My discoveries}

\subsection{Coupled monthly variability of the South Asian and East Asian monsoons in July-August}

\subsection{Coordinated changes between the Meiyu Front and tropospheric jet}

\section{Potential Future Work}

\subsection{Can Known Modes of Variability Constrain Future Rainfall Change?}

In terms of the practical consequences of global warming, changes in mean and extreme rainfall are surely the largest impact. Two immediate consequences include impacts on agriculture, and changes in the frequency of drought and flooding

The principal finding of the previous decade has been that rainfall does not increase with Clausius-Clapeyron scaling (7\%/$^{\circ}$K), but at a slower rate of 2-3\%/$^{\circ}$K.

\subsection{Changes in 21st century moisture transport from the Bay of Bengal to central China}

Since we have argued that the South Asian monsoon and East Asian monsoon are linked by the transport of moisture across the Yunnan Plateau, then any potential changes in this transport could strongly influence the future mean state of the Asian monsoon in July and August

\subsection{The Key Role of Low Orography in Monsoon summer Hydrology}

A central finding of this study is that low orography has a major impact on the distribution of mean rainfall. In addition, the local response to broader modes of variability can feature sharp regime transitions. This resembles past work on convective margins by authors such as \cite{Lintner2007}.

\subsection{The future of the ``South Flood-North Drought''}

Combined with the trend of desertification of northern China, a decrease in precipitation in northern China has the potential of changing a wide swath of agricultural heartland into inarable terrain.