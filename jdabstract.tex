% (This file is included by thesis.tex; you do not latex it by itself.)

\begin{abstract}

% The text of the abstract goes here.  If you need to use a \section
% command you will need to use \section*, \subsection*, etc. so that
% you don't get any numbering.  You probably won't be using any of
% these commands in the abstract anyway.

The Asian summer monsoon supplies around 3 billion people with much of their yearly supply of freshwater, necessary for human consumption as well as in agriculture and industry. In many regions, particularly along the Ganges River in India and in northern China, use of freshwater far exceeds natural recharge rates. Given the high population density of these regions, a substantial fraction of Asia's population is therefore critically sensitive to interannual changes in the supply of freshwater by the monsoon, as well as potential future change under 21st century warming. This dissertation focuses on understanding the atmospheric dynamics of the leading mode of July-August Asian Monsoon rainfall variability, which links two major subsystems: the South Asian and East Asian monsoons.

In summer, two distinct rainfall r\'{e}gimes are observed: Convective storms over India, Bangladesh and Nepal (the South Asian monsoon), and frontal rainfall over China, Japan and Korea (the East Asian monsoon). In addition, the Himalayas and other orography, including the Arakan Mountains, Ghats and Yunnan Plateau, create smaller precipitation domains separated by sharp gradients. My research has revealed a previously unrecognized mode of continental precipitation variability that spans both South and East Asia during July and August. A dipole between the Himalayan Foothills and the ��Monsoon Zone�� of central India dominates July-August interannual variability in South Asia, and is also associated in East Asia with a tripole between the Yangtze Corridor and North and South China. By performing lag-lead correlation of rainfall, I show that this covariation does not correspond to the spatial pattern of July-August storm tracks. Instead, I hypothesize that interannual change in the strength of moisture transport from the Bay of Bengal to the Yangtze Corridor across the northern Yunnan Plateau induces widespread precipitation anomalies. Abundant moisture transport along this route requires both cyclonic monsoon circulation over India and sufficient heating over the Bay of Bengal, which occurs only during July and August, as I also show by analyzing existing runs with a zoomed, nested version of the LMDZ5 model nudged to reanalysis.

In China, a growing body of work identifies a r\'{e}gime shift in rainfall occurring in the late 1970s known as the ``South Flood-North Drought.'' However, this phenomenon has not previously been described in terms of the complex seasonal cycle of the East Asian monsoon. During the peak interval of rainfall, known as Meiyu season, a persistent but meandering front (the ``��Meiyu Front''��) delivers strong bouts of rainfall over a narrow latitude band. The preferred latitude of this front shifts throughout the year and shifts abruptly from May to July. I have developed an image processing algorithm, the Rainband Detection Algorithm (RDA), that produces a 57-year (1951-2007) daily catalog of frontal rainfall events in China, reporting latitude, intensity and zonal extent. This result allows for a quantitative assessment of Meiyu change between 1951-1979 and 1980-2007, which has never previously been performed on an event-by-event basis. I find that the greatest change in frontal behavior has occurred during May, when events have strongly decreased over the Yangtze River valley. In addition, a statistically significant southward shift has occurred in July-August Meiyu events. The rainfall changes associated with South Flood-North Drought can be attributed specifically to changes in banded rainfall.  A suite of alternative simple metrics of China rainfall cannot capture all of the characteristics of the South Flood-North Drought in one succinct measure. Another distinct change occurred between 1979-1993 and 1994-2007 featuring intensification of Meiyu rainbands, possibly reflecting a different causal mechanism. Our RDA catalog is available to other East Asian monsoon researchers.

The leading Asian monsoon mode described in Chapter 2 (hereafter All-Asia EOF1) not only describes a key mode of Asian monsoon variability, but is also associated with climate anomalies across the entire Northern Hemisphere. We find an association between strong All-Asia EOF1 years and the Circumglobal Teleconnection (CGT), a high-latitude standing wave pattern with wavenumber 5 or 6 implicated in Northern Hemisphere heat wave events, which we further investigate through compositing with JRA-55 reanalysis. JRA-55 captures changes in water vapor transport across the Yunnan Plateau associated with All-Asia EOF1 as hypothesized in Chapter 2, and suggests that they correspond to an anomalous band of diabatic heating of amplitude 100 W m$^{-2}$. We propose that this heating can force the CGT and participates in the phase-locking of the CGT, which is further amplified by the configuration of high topography. Over China and East Asia, All-Asia EOF1 is manifested as a meridional jet shift. Using the RDA catalog from Chapter 3, we test whether rainfall shifts across the entire East Asian monsoon also correspond to jet changes. We find that the South Flood-North Drought robustly corresponds to a southward shift in the East Asian tropospheric jet. Thus, although IPCC5 projections of the \nth{21}-century Asian monsoon remain inconclusive, future changes may be more predictable by further studying the effects of global warming on Bay of Bengal moisture transport, the CGT and the East Asian jet.

\end{abstract}
